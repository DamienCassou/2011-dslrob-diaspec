%!TEX root=dslrob.tex

\section{Discussing}
\label{sec:discussing}

\diaspec{} decomposes the development of an application into two well
defined stages: a design stage for which \diaspec{} provides a
domain-specific design language and an implementation stage for which
\diaspec{} provides a design-specific programming framework. With the
design language and SCC architectural pattern, a developer is guided
in creating components with a single responsibility each, thus
enhancing reuse. An application design also explicits interactions
between components making the runtime behavior easier to understand.
With the programming framework dedicated to the design, a developer is
guided in creating an implementation for each components.
Indeed, the generated programming framework takes care of the control
loop of the application as well as all interactions between the
components. As a result, a developer can focus on implementing the
high-level application logic, letting the framework handle the
details. Moreover, the programming framework provides all necessary
pieces of required information directly as parameters to the abstract
methods. This reduces the amount of documentation required to start
using the programming framework.

In the previous sections we saw that \diaspec{} can be used to design,
implement and deploy a robotics application for a widely used
middleware. In the following we discuss various problems we have met
while applying a\diaspec{} in a robotics setting.

\subsection{\diaspec{} Dynamicity}

\diaspec{} has been created to handle appearing and disappearing
entities at runtime. This requires additional code that should not be
necessary in a robotics settings where most, if not all, entities are
known at deployment time. For example, to receive information from
entity sources, context operators have to subscribe to each source of
interest:

\begin{lstlisting}[language=java]
@Override
protected void postInitialize() {
  discoverExplorationForSubscribe.all().subscribeTwist();
  discoverModeSelectorForSubscribe.all().subscribeMode();
}
\end{lstlisting}

This method has to be implemented in the \ct{Motion} Java class (for
the \ct{Motion} context operator) to let it receive information from
the \ct{Exploration} and \ct{ModeSelector} entities. This code could
be inferred automatically from the description of the \ct{Motion}
context operator and pushed inside the generated programming
framework. However, in a multi-robots settings, where a robot can
discover services provided by nearby robots, the \diaspec{} entity
discovery and subscription mechanisms still could be useful. The
\diaspec{} design language could be extended to let a developer
declare which entities are known at deployment time and which ones
should be discovered at runtime. The compiler could then leverage this
additional information to generate the necessary code in the
programming framework.

\subsection{Data Type Reuse}

\diaspec{} allows the definition of new types (structures and
enumerations) as well as the importation of existing Java types. Very
often, middleware such as ROS come with their own data types. The
developer must then choose to reuse the data types coming from the
middleware or define new ones. Using the middleware data types can be
particularly useful as these data types can be complex such as the ROS
``twist'' data type. This is the solution we use for the case study
and the \ct{Twist} data type as is illustrated by the use of the
\ct{import} keyword in Listing~\ref{listing:design},
line~\ref{design:import}. However, choosing reuse of data types from a
middleware tightly couples the application with this middleware and
thus prevents potential for reuse of this application with other
middleware. Another solution is to develop new data types in
\diaspec{}. This makes the application independent from any underlying
middleware. However, this requires conversion code at the boundaries
of the application where communication with the middleware is
required. For example, it could be possible to define the \ct{Twist}
data type within the design with some code like this:

\begin{lstlisting}[language=diaspec, numbers=none]
structure Vector3 { x as Float; y as Float; z as Float; }
structure Twist { linear as Vector3; angular as Vector3; }
\end{lstlisting}

Then, an implementation of the \ct{Wheel} entity would have to convert
from the \diaspec{} \ct{Twist} type to the ROS \ct{Twist} type:

\begin{lstlisting}[language=java, numbers=none]
@Override
protected void setTwist(Twist twist) throws Exception {
  publisher.publish(convert(twist));
}
private org.ros.message.geometry_msgs.Twist
                                   convert(Twist twist) {
  org.ros.message.geometry_msgs.Twist rosTwist;
  rosTwist = new org.ros.message.geometry_msgs.Twist();
  rosTwist.angular = convert(twist.getAngular());
  rosTwist.linear = convert(twist.getLinear());
  return rosTwist;
}
private org.ros.message.geometry_msgs.Vector3
                            convert(Vector3 vector) {...}
\end{lstlisting}

This solution makes the code harder to read and maintain. Moreover,
similar code has to be duplicated everywhere in the application where
a conversion is required. An intermediate solution is to develop new
data types in Java. This solution can embed required conversions in
the data type itself to avoid duplication. The resulting code is still
harder to read than the first one however.

\subsection{Decomposition grain}

During the development of the case study we noticed that following the
SCC architectural pattern and the steps proposed in
Section~\ref{sec:designing} resulted in fine grained components,
promoting reuse. It is however important that the developer pays
attention not to create too fine grained components which would make
the runtime behavior hard to understand and debug. Indeed, because the
generated programming framework handles the interactions between the
components, debugging very fine grained components requires stepping
often into the generated programming framework. This is cumbersome and
should not be needed. A possible addition to \diaspec{} could involve
a dedicated debugger which would let the developer debug his
application without stepping into the generated programming framework.

% LocalWords:  SCC hoc

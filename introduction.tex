%!TEX root=dslrob.tex
\section{Introduction}

A Sense/Compute/Control (SCC) application is one that interacts with
the environment~\cite{Tayl09a}. The SCC architectural pattern
guides the description of SCC applications and involves four kinds of
components, organized into layers~\cite{Cass11a,Edwar09a}: (1)
\emph{sensors} at the bottom, which obtain information about the
environment; (2) then \emph{context operators}, which process this
information; (3) then \emph{control operators}, which use this refined
information to control (4) \emph{actuators} at the top, which finally
impact the environment. A robotics application is a kind of SCC
application where the environment is composed of a robot
(sensors/actuators/body, control architecture, \etc{}) and the robot's
neighborhood (the walls, ground, people, \etc{})~\cite{Sicil08a}. As
noticed by Taylor \etal{}~\cite{Tayl09a}, the Sense/Plan/Act
architecture~\cite{Sicil08a}, widely used in robotics, closely
resembles the SCC architectural pattern.

\diaspec{} is a domain-specific design language dedicated to
describing SCC applications~\cite{Cass09b,Cass11a}. From such a design
description, the \diaspec{} compiler produces a dedicated Java
programming framework that is both \emph{prescriptive} and
\emph{restrictive}: it is prescriptive in the sense that it guides the
developer, and it is restrictive in the sense that it limits the
developer to what the design description allows. By separating
application logic (implemented by the developers) and runtime support
(generated in the programming framework), \diaspec{} facilitates the
design, implementation and evolution of SCC applications.

%\damien{Talk about problems in the robotics domain: ad-hoc solutions, hard to reuse, hard to adapt to new environments...}

\subsection*{Contributions}

Our contributions are as follows:

\begin{itemize}
\item \emph{A report} on an experiment of designing and implementing a
  standard robotics application in the SCC architectural pattern with  the \diaspec{} domain-specific design language and framework
  (Sections~\ref{sec:designing} and~\ref{sec:implementing}). This
  report includes detailed instructions and guidelines to allow
  further experiments.
\item \emph{A discussion} of the benefits and problems of using
  \diaspec{} in a robotics setting (Section~\ref{sec:discussing}).
  This discussion includes a list of changes to \diaspec{} that would
  make it a better framework for developing new robotics applications.
\end{itemize}

We finally highlight some related works and conclude in
sections~\ref{sec:related} and~\ref{sec:conclusion}.

%%% LocalWords:  SCC




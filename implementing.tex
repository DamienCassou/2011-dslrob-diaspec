%!TEX root=dslrob.tex

\section{Implementing a robotics application}
\label{sec:implementing}

The \diaspec{} compiler generates a programming framework with respect
to a set of declarations for entity classes, context operators and
control operators. For each component description (entity or operator)
the compiler generates an abstract class. The abstract methods in this
class represent code to be provided by the developer, to allow him to
program the application logic (\eg{} to trigger an entity action). 

Implementing a \diaspec{} component is done by \textit{sub-classing}
the corresponding generated abstract class. In doing so, the developer
is required to implement each abstract method. The developer writes
the application code in subclasses, not in the generated abstract
classes. This strategy contrasts with generating incomplete source
code, to be filled by the developer. As a result, in our approach, one
can change the \diaspec{} description and generate a new programming
framework without overriding the developer's code. The mismatches
between the existing code and the new programming framework are
revealed by the Java compiler. To facilitate the implementation
process, most Java IDEs are capable of generating class templates
based on super abstract classes.

In this section, we give an overview of how to implement some parts of
the case study. For a more detailed description, we refer to our
previous works~\cite{Cass09b,Cass11a,Cass11b}.

\subsection{Implementing an operator}

For each context or control operator, a dedicated abstract class is
generated in the programming framework. For each interaction contract
of this operator, the generated abstract class contains an
\emph{abstract method} and a corresponding \emph{calling method}. The
abstract method is to be implemented by the developer while the
calling method is used by the framework to call the implementation of
the abstract method with the expected arguments.

The signature of each abstract method is directly derived from the
interaction contract: the name of the method is derived from the
activation condition, the return type is derived from the reaction,
and the parameters are derived from the activation condition and the
reaction.

Listing~\ref{listing:contextop-implem} presents a possible Java
implementation of the \ct{RandomMotion} context operator. The
\ct{onObstacleDetection} method is declared abstract in the
\ct{AbstractRandomMotion} generated super class.

\lstinputlisting%
[float,language=java,%
caption={A developer-supplied Java implementation of the
  \texttt{Random\-Motion} context operator described in
  Listing~\ref{listing:design}. The \texttt{Abstract\-Random\-Motion}
  super class is automatically generated},%
label={listing:contextop-implem}]%
{code/RandomMotion.java}

Because an operator only manipulates input sources to produce a
result, its implementation is independent of any robotics software
framework. This facilitates operator reuse for different applications
and robots.

\subsection{Implementing an entity}

Contrary to operators which are dedicated to the application logic, an
entity is at the border between the application and its environment
(\eg{} the middleware and robot hardware). Implementing an entity thus
requires some knowledge of the underlying hardware or middleware.

Listing~\ref{listing:laserscan-implem} presents a possible Java
implementation of the \ct{LaserScan} entity class for the ROS
middleware. When the middleware publishes a new laser scan message,
this message is automatically received by the \ct{RosLaserScan}
instance through the ROS \ct{MessageListener} interface.

\lstinputlisting%
[float,language=java,%
caption={A developer-supplied Java implementation of the
  \texttt{LaserScan} entity class described in
  Listing~\ref{listing:design}, line~\ref{design:laserscan-b}. The
  \texttt{Abstract\-Laser\-Scan} super class is automatically
  generated},%
label={listing:laserscan-implem}]%
{code/RosLaserScan.java}

Listing~\ref{listing:light-implem} presents a possible Java
implementation of the \ct{Light} entity class for the ROS middleware.
The constructor receives a ROS \ct{publisher} as a parameter which
allows the entity implementation to send commands to the robot through
the middleware.

\lstinputlisting%
[float,language=java,%
caption={A developer-supplied Java implementation of the
  \texttt{Light} entity class described in
  Listing~\ref{listing:design}, line~\ref{design:light-b}. The
  \texttt{Abstract\-Light} super class is automatically generated},%
label={listing:light-implem}]%
{code/RosLight.java}

\subsection{Deploying an application}

Deploying an application requires writing a deployment script in Java.
To do this, a developer creates a new Java class by sub-classing the
abstract class \ct{MainDeploy} generated in the programming framework.
By doing so the developer is required to implement one abstract method
per operator, to call the \ct{add()} method to register entity
instances, and to call the \ct{deployAll()} method to trigger the
deployment. The ROS middleware requires an implementation of the
\ct{NodeMain} interface. An extract of the deployment script for the
case study application is shown in Listing~\ref{listing:deploy}.

\lstinputlisting%
[float,language=java,%
caption={An extract of a developer-supplied Java deployment script for
  the case-study application},%
label={listing:deploy}]%
{code/Deploy.java}

\serge{Insérer une photo d'écran de la simulation ?}

In this section we saw how to implement a robotics application on top
of a programing framework generated by the \diaspec{} compiler. This
programming framework calls developer's code when necessary and make
the development easy by passing everything the developer needs as a
parameter to abstract methods. In the next section we discuss the
benefits and problems of using \diaspec{} in a robotics setting.

\section{Designing a robotics application}
\label{sec:designing}

In this section we first explain how to decompose a robotics
application in \diaspec{} component types. Then we present a case
study robotics application that we use as an example of how to
describe a robotics application with \diaspec{}.

\subsection{Decomposing}

Designing an application with \diaspec{} requires a decomposition in
layers. Each layer corresponds to a separate class of components:

\begin{itemize}
\item A \emph{sensor} sends information sensed from the environment to
  the context operator layer through data \emph{sources}. A sensor can
  both push data to context operators and respond to context operator
  requests. We use the term ``sensor'' both for entities that actively
  retrieve information from the environment, such as system probes,
  and entities that store information previously collected from the
  environment, such as databases.
\item A \emph{context operator} refines (aggregates and interprets)
  the information given by the sensors. Context operators can push
  data to other context operators and to control operators. Context
  operators can also respond to requests from parent context
  operators.
\item A \emph{control operator} transforms the information given by
  the context operators into orders for the actuators.
\item An \emph{actuator} triggers actions on the environment.
\end{itemize}

The following gives some guidelines to decompose a robotics
application into components.

\paragraph{Reusing existing components}
In the presence of a previous application developed with \diaspec{},
it is possible and advisable to reuse as much components as possible.
Depending on the amount of reused components, this can have a huge
impact on the application of the other guidelines.

\paragraph{Listing capabilities}
Each robot comes with its own set of capabilities (\eg{} sensing
motion and projecting light). These capabilities should be mapped to
sensor sources and actuator actions. Related sources and actions
should then be grouped inside \emph{entity classes} (\eg{} a camera
providing a picture source and zooming action). Beside sources and
actions, an entity class may also have \emph{attributes} to
characterize its instances (\eg{} resolution, accuracy and status). In
the presence of a high-level software framework (such as ROS), it can
be useful to also map capabilities of the framework into sources and
actions (\eg{} a mapping capability).

\paragraph{Identifying main context operators}
The next step of the decomposition in components is the identification
of the main high-level pieces of information required by the
application. These pieces of information are represented as context
operators and directly used as input to control operators.

\paragraph{Decomposing into lower-level pieces}
Then, lower-level context operators must be identified to act as input
sources for the higher-level ones. This decomposition is typically
done in several steps, each step slightly lowering the level of
previously identified context operators. This decomposition ends when
each identified context operator can directly take its input from a
set of sensor sources.

\paragraph{Identifying control operators}
From the high-level context operators, it is then necessary to derive
a set of control operators that are going to send orders to actuators
based on this information. Because the code of a control operator can
not be reused in another part of the application, it is important that
this code is as simple as possible. If there is opportunity for reuse,
the code should be moved to a new context operator.

\subsection{Case Study}

As a running example, we present an application that is typical of the
robotics domain.

\subsection{Describing with \diaspec{}}

Once the application is decomposed using the different component types,
the transcription to the \diaspec{} design language is
straightforward. Listing~\ref{listing:design} gives the transcription
of the case study.

\lstinputlisting[float,language=diaspec,breakatwhitespace=true%
,caption={Description of the robotics application with the \diaspec{} language}%
,label={listing:design}]%
{code/design.diaspec}

%%% Local Variables:
%%% mode: latex
%%% coding: utf-8
%%% TeX-master: "dslrob"
%%% TeX-PDF-mode: t
%%% ispell-local-dictionary: "english"
%%% End:

%%%%%%%%%%%%%%%%%%%%%%%%%%%%%%%%%%%%%%%%%%%%%%%%%%%%%%%%%%%%%%%%%%%%%%%%%%%%%%%%
%2345678901234567890123456789012345678901234567890123456789012345678901234567890
%        1         2         3         4         5         6         7         8

\documentclass[letterpaper, 10 pt, conference]{ieeeconf}  % Comment this line out
                                                          % if you need a4paper
%\documentclass[a4paper, 10pt, conference]{ieeeconf}      % Use this line for a4
                                                          % paper

\IEEEoverridecommandlockouts                              % This command is only
                                                          % needed if you want to
                                                          % use the \thanks command
\overrideIEEEmargins
% See the \addtolength command later in the file to balance the column lengths
% on the last page of the document

\usepackage[pdftex]{graphicx}
\graphicspath{{resources/}}

% The following packages can be found on http:\\www.ctan.org
%\usepackage{graphics} % for pdf, bitmapped graphics files
%\usepackage{epsfig} % for postscript graphics files
%\usepackage{mathptmx} % assumes new font selection scheme installed
%\usepackage{times} % assumes new font selection scheme installed
%\usepackage{amsmath} % assumes amsmath package installed
%\usepackage{amssymb}  % assumes amsmath package installed
\usepackage[T1]{fontenc}
\usepackage[utf8]{inputenc}
\usepackage{xspace}  
\usepackage{cite}  

\newcommand{\diaspec}{Dia\-Spec\xspace}
\newcommand{\ie}{\emph{i.e.,}}
\newcommand{\eg}{\emph{e.g.,}}
\newcommand{\etc}{\emph{etc}}
\newcommand{\etal}{\emph{et al.}}

\title{Using the \diaspec{} design language and compiler to develop
  robotics systems}

\author{%
  \parbox{3 in}{\centering Damien Cassou\\
    Software Architecture Group, HPI\\
    University of Potsdam, Germany\\%
    {\tt\small damien.cassou@hpi.uni-potsdam.de}}
  \hspace*{ 0.5 in}
  \parbox{3 in}{ \centering Serge Stinckwich and Pierrick Koch\\
    IRD, UMMISCO\\
    Hanoi, Vietnam\\
    {\tt\small serge.stinckwich@ird.fr}}
}

\begin{document}

\maketitle
\thispagestyle{empty}
\pagestyle{empty}


%%%%%%%%%%%%%%%%%%%%%%%%%%%%%%%%%%%%%%%%%%%%%%%%%%%%%%%%%%%%%%%%%%%%%%%%%%%%%%%%
\begin{abstract}

  A Sense/Compute/Control (SCC) application is one that interacts with
  the physical environment. Such applications are pervasive in domains
  such as building automation, assisted living, and autonomic
  computing. Developing an SCC application is complex because the
  implementation must address both the interaction with the
  environment and the application logic, because any evolution in the
  environment must be reflected in the implementation of the
  application, and because correctness is essential, as effects on the
  physical environment can have irreversible consequences.

  The SCC architectural pattern and the \diaspec{} design language
  propose a framework to guide the design of such applications. From a
  design description in \diaspec{}, the \diaspec{} compiler is capable
  of generating a programming framework that guide the developer in
  implementing the design and that provide runtime support. In this
  paper, we report on an experiment using \diaspec{} (both the design
  language and compiler) to develop a robotics application. We present
  how using \diaspec{} has enabled us to develop a robotics
  application that can be adapted easily.

\end{abstract}

%!TEX root=dslrob.tex
\section{Introduction}

A Sense/Compute/Control (SCC) application is one that interacts with
the environment~\cite{Tayl09a}. The SCC architectural pattern
guides the description of SCC applications and involves four kinds of
components, organized into layers~\cite{Cass11a,Edwar09a}: (1)
\emph{sensors} at the bottom, which obtain information about the
environment; (2) then \emph{context operators}, which process this
information; (3) then \emph{control operators}, which use this refined
information to control (4) \emph{actuators} at the top, which finally
impact the environment. A robotics application is a kind of SCC
application where the environment is composed of a robot
(sensors/actuators/body, control architecture, \etc{}) and the robot's
neighborhood (the walls, ground, people, \etc{})~\cite{Sicil08a}. As
noticed by Taylor \etal{}~\cite{Tayl09a}, the Sense/Plan/Act
architecture~\cite{Sicil08a}, widely used in robotics, closely
resembles the SCC architectural pattern.

\diaspec{} is a domain-specific design language dedicated to
describing SCC applications~\cite{Cass09b,Cass11a}. From such a design
description, the \diaspec{} compiler produces a dedicated Java
programming framework that is both \emph{prescriptive} and
\emph{restrictive}: it is prescriptive in the sense that it guides the
developer, and it is restrictive in the sense that it limits the
developer to what the design description allows. By separating
application logic (implemented by the developers) and runtime support
(generated in the programming framework), \diaspec{} facilitates the
design, implementation and evolution of SCC applications.

%\damien{Talk about problems in the robotics domain: ad-hoc solutions, hard to reuse, hard to adapt to new environments...}

\subsection*{Contributions}

Our contributions are as follows:

\begin{itemize}
\item \emph{A report} on an experiment of designing and implementing a
  standard robotics application in the SCC architectural pattern with
  the \diaspec{} domain-specific design language and framework
  (Sections~\ref{sec:designing} and~\ref{sec:implementing}). This
  report includes detailed instructions and guidelines to allow
  further experiments.
\item \emph{A discussion} of the benefits and problems of using
  \diaspec{} in a robotics setting (Section~\ref{sec:discussing}).
  This discussion includes a list of changes to \diaspec{} that would
  make it a better framework for developing new robotics applications.
\end{itemize}

We finally highlight some related works and conclude in
sections~\ref{sec:related} and~\ref{sec:conclusion}.

%%% LocalWords:  SCC





\bibliographystyle{plain}
\bibliography{dslrob}

\end{document}

%%% Local Variables: 
%%% mode: latex
%%% TeX-master: t
%%% TeX-PDF-mode: t
%%% coding: utf-8
%%% ispell-local-dictionary: "english"
%%% End:

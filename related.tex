%!TEX root=dslrob.tex

\section{Related Work}
\label{sec:related}

Several software engineering approaches have been proposed to lower
the complexity of robotics systems~\cite{Brug07a}. Among them we can cite:

\begin{enumerate}
\item {\bf Numerous middlewares and software frameworks} have been proposed to
support the implementation of robotics applications (\eg{}
CLARATy~\cite{Claraty}, ROS~\cite{ROS} and Player/Stage~\cite{Coll05a}).
Such approaches attempt to cover as much of the robotics domain as
possible in a single programming framework. This strategy often leads
to large APIs, providing little guidance to the developer and
requiring boilerplate code to customize the programming framework to
the characteristics of the application. In contrast, a
\diaspec{}-generated programming framework specifically targets one
application, limiting the API to methods of interest to the
developers. Our code generator could potentially target these
middlewares thus leveraging existing work and completely hiding their
intricacies from the developer.

\item {\bf Component-based software engineering for robotics~\cite{Brug07b} and
model-driven software engineering for Robotics} (e.g., OMG
RTC~\cite{OMGRTC}, SmartSoft~\cite{Schl09a}). All these approaches
apply and tailor general-purpose and established principles of
lowering complexity to robotics needs and come up with domain-specific
extensions.

\item {\bf Domain-specific languages for the orchestration of robotics component software} (e.g \emph{Smach}~\cite{Boren10a}, \emph{SmartTCL}~\cite{Stec11a}). \emph{Smach} is a Python embedded DSL based on hierarchical concurrent state machines for building complex robot behaviors from primitive ones. \emph{Smach} is tightly coupled with ROS. Smach only allow static compositions of behaviors and could not adapt compositions to new situations during execution.
SmartTCL (Smart Task Coordination Language) is an extension of CommonLisp that is used to do online dynamic reconfiguration of the software components involved in a robot: knowledge bases, simulation engines, symbolic task planners, models and low-level hardware. At design time, the developer defines execution variants that robot operates at runtime. In order to lower robotics inherent complexity, analysis and simulation tools could also be used at runtime to determine pending execution steps with specific parametrisation before the robot effectively execute them.

Unlike the two previous DSL, Diaspec allows as we have previously shown a natural decomposition of applications according to the pattern SCC, which will therefore guide the work of the developer. The main lack of Diaspec compared to SmartTCL is that it does not allow a dynamic recomposition of software components at runtime.

\end{enumerate}
Add to reference: David Kortenkamp, Reid Simmons, Chapter 8 - Robotic Systems Architectures and Programming, Handbook of Robotics, Springer Verlag.~\cite{Kort08a}

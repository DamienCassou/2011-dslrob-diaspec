%!TEX root=dslrob.tex

\section{Related Work}
\label{sec:related}

Several software engineering approaches have been proposed to lower
the complexity of robotics systems~\cite{Brug07a}.

Numerous middlewares and software frameworks have been proposed to
support the implementation of robotics applications (\eg{}
CLARATy~\cite{Claraty}, ROS~\cite{ROS} and Player~\cite{Coll05a}).
Such approaches attempt to cover as much of the robotics domain as
possible in a single programming framework. This strategy often leads
to large APIs, providing little guidance to the developer and
requiring boilerplate code to customize the programming framework to
the characteristics of the application. In contrast, a
\diaspec{}-generated programming framework specifically targets one
application, limiting the API to methods of interest to the
developers. Our code generator could potentially target these
middlewares thus leveraging existing work and completely hiding their
intricacies from the developer.

Component-based software engineering for robotics~\cite{Brug07b} and
model-driven software engineering for Robotics (e.g., OMG
RTC~\cite{OMGRTC}, SmartSoft~\cite{Schl09a}). All these approaches
apply and tailor general-purpose and established principles of
lowering complexity to robotics needs and come up with domain-specific
extensions.

Domain-specific languages for robotics (e.g Smach~\footnote{\url{http://www.ros.org/wiki/smach} Reference to add: J. Boren and S. Cousins, “The SMACH High-Level Executive [ROS News],” IEEE RAM, vol. 17, no. 4, pp. 18–20, 2010. }, SmartTCL~\footnote{Reference to add: Andreas Steck and Christian Schlegel, "Managing Execution Variants in Task Coordination by Exploiting Design-Time Models at Run-Time", in Proc. IEEE/RSJ Int. Conf. on Intelligent Robots and Systems (IROS), San Francisco, California, USA, September 2011. (to appear)}, ...).

Smach is a Python embedded DSL based on hierarchical concurrent state machines for composing complex robot behaviors from primitive ones.

SmartTCL (Smart Task Coordination Language) is an extension of CommonLisp that is used to do online dynamic reconfiguration of the software components involved in a robot: knowledge bases, simulation engines, symbolic task planners, models and low-level hardware. At design time, the developer defines execution variants that robot operates at runtime. In order to lower robotics inherent complexity, analysis and simulation tools could also be used at runtime to determine pending execution steps with specific parametrisation before the robot effectively execute them.

Dire qu'on a une approche dédiée.

Add to reference: David Kortenkamp, Reid Simmons, Chapter 8 - Robotic Systems Architectures and Programming, Handbook of Robotics, Springer Verlag.

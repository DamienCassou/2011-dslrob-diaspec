%!TEX root=dslrob.tex

\section{Related Work}
\label{sec:related}

Several software engineering approaches have been proposed to lower
the complexity of robotics systems~\cite{Brug07a}.

\paragraph*{Middleware and Software Frameworks}

Numerous middleware and software frameworks have been proposed to
support the implementation of robotics applications (\eg{}
CLARATy~\cite{Claraty}, ROS~\cite{ROS} and
Player/Stage~\cite{Coll05a}). Such approaches attempt to cover as much
of the robotics domain as possible in a single programming framework.
This strategy often leads to large APIs, providing little guidance to
the developer and requiring boilerplate code to customize the
programming framework to the characteristics of the application. In
contrast, a \diaspec{}-generated programming framework specifically
targets one application, limiting the API to methods of interest to
the developers. Our code generator could potentially target these
middleware thus leveraging existing work and completely hiding their
intricacies from the developer.

\paragraph*{Component-Based and Model-Driven Software Engineering}

Component-Based Software Engineering for robotics
(\eg{}~\cite{Brug07b}) and Model-Driven Engineering for robotics
(\eg{} OMG RTC~\cite{OMGRTC}, SmartSoft~\cite{Schl09a}) relies on
general-purpose notations such as UML to model domain-specific
concerns. By using general-purpose and established notations, these
approaches leverage existing knowledge from developers and existing
tools. Even though such approaches propose a conceptual framework for
developing robotics applications, they only provide the user with
generic tools. For example, these approaches require developers to
directly manipulate UML diagrams, which become ``enormous, ambiguous
and unwieldy''~\cite{Picek08a}. In contrast, \diaspec{} abstracts away
such technologies, limiting the amount of expertise required from the
developers.

\paragraph*{Domain-Specific Languages}

\emph{Smach} is a Python embedded DSL based on hierarchical concurrent
state machines for building complex robot behavior from primitive
ones~\cite{Boren10a}. \emph{Smach} is tightly coupled with ROS, allows
only static compositions of behavior and can not adapt compositions to
new situations during execution. SmartTCL (Smart Task Coordination
Language) is an extension of Common Lisp that is used to do on line
dynamic reconfiguration of the software components involved in a
robot~\cite{Stec11a}: knowledge bases, simulation engines, symbolic
task planners, models and low-level hardware. At design time, the
developer defines execution variants that robot operates at runtime.
In order to lower robotics inherent complexity, analysis and
simulation tools could also be used at runtime to determine pending
execution steps with specific parametrisation before the robot
effectively execute them. Unlike these DSLs, \diaspec{} allows a
natural decomposition of applications according to the SCC
architectural pattern, guiding the work of the developer. Compared to
SmartTCL \diaspec{} lacks the ability to recompose the components at
runtime.

% Add to reference: David Kortenkamp, Reid Simmons, Chapter 8 - Robotic
% Systems Architectures and Programming, Handbook of Robotics, Springer
% Verlag.~\cite{Kort08a}

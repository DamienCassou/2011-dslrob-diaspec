%!TEX root=dslrob.tex

\section{Conclusions and future works}
\label{sec:conclusion}

In this paper, we have proposed to use \diaspec{}, a domain-specific
design language for Sense/Compute/Control applications, in a robotics
setting. We have shown how this language allows a developer to
structure an application in fine-grained and reusable components by
following the SCC architectural pattern. Developing a complex
application shows the benefits of such an approach regarding reuse of
existing software components and diminution of complexity for the
developer. We have also highlighted  problems we have met
during the development of a standard robotics application.

Being able to adapt a robotics system to different capabilities and
resources is a key issue in software engineering for robotics. For
example, a robot can perform two similar missions differently with
different resources. Our approach facilitates changes to a robotics
system by making explicit the software components and their
interactions. However supporting static adaptation is insufficient in
a robotics setting as robots need to dynamically adapt to resource
evolutions (\eg{} failures and environment) while performing their
tasks. \emph{Resource-adaptive architectures} address dynamic
adaptations. However, such architectures are ad hoc solutions that
developers can hardly reuse and scale. Therefore, an ideal robot
control architecture should be \emph{resource-adapting}, \ie{} an
architecture that explicitly manages and represents
resources~\cite{Bour07a}. The main perspective of this work is to
introduce such dynamic variability inside \diaspec{}.

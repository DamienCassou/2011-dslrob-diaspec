%!TEX root=dslrob.tex

\section{Conclusions and future works}
\label{sec:conclusion}

We propose to use DiaSpec, a domain-specific design language, designed at the beginning for pervasive applications, in the new context of robotics applications. We show how this language allow the developers to decompose an application in finer grains by following the SCC architectural pattern.
Developing a quite complex application shows the benefits of such an approach in terms of reusing existing software and lowering the burden of the complexity for the developers. Our experiment give us also some ideas on how we could adapt DiaSpec to the robotics specific concerns.

To be able to adapt execution at runtime is very important in a robotic context. One possible change in the execution context is the use of robots with different capabilities and resources. The same missions can be performed differently (reactively or in a more deliberative way) according to robots resources. Software architecture should be adapted statically between two robotics deployments. Robot’s control architecture should not only bear static adaptation, but it should also allow dynamic adaptation. Robots controlled by such an architecture has to react to the evolution of their environment (evolution of resources, robot failures, ...) in order to be as efficient as possible. At the end, we should be able to design (1) \emph{resource-adapted systems}, that are pre-configure to a specific domain, but also (2) \emph{resource-adaptive systems}, that are able to react to resource modifications, and (3) \emph{resource-adapting systems}, that explicitly manage and represent resources\footnote{N. Bouraqadi and S. Stinckwich. Towards an adaptive robot control architecture. In 2nd Workshop on "Control Architectures of Robots: from models to execution on distributed control architectures”, pages 135–149, Paris, May 2007.}. The main perspective of this work is to introduce such dynamic variabilites in DiaSpec.
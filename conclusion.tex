%!TEX root=dslrob.tex

\section{Conclusions and future works}
\label{sec:conclusion}

We propose to use DiaSpec, a domain-specific design language, already used in the context of pervasive applications, in the new context of robotics applications. We show how this language allow the developers to decompose an application in finer grains by following the SCC architectural pattern.
Developing a quite complex application shows the benefits of such an approach in terms of reusing existing software and lowering the burden of the complexity for the developers. Our experiment give us also some ideas on how we could adapt DiaSpec to the robotics specific concerns.
The main perspective of this work is to introduce some dynamic aspects in DiaSpec in order to adapt the robot software architecture to new situations and to manage ressources.